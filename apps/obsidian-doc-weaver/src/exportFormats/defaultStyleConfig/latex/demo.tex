% Demo: Obsidian-style Callout and multi-column layout in LaTeX
% Build: pdflatex demo.tex  (for Chinese use: xelatex + ctex)
% Requires: multicol, geometry, enumitem, lipsum
\documentclass[11pt]{article}
\usepackage[utf8]{inputenc}
\usepackage[T1]{fontenc}
\usepackage{geometry}
\geometry{a4paper, margin=2cm}

% 把 callout 的样式与图标定义集中在独立 sty 中,便于复用和维护。
\usepackage{DW_styles}
\usepackage{multicol}
\usepackage{paracol}
\usepackage{enumitem}
\usepackage{lipsum}
\usepackage{graphicx}

% ========== 正文 ==========
\begin{document}

\title{Callout \& Multi-column Demo}
\author{Obsidian-format-toolkit}
\maketitle

\section{Callout examples}

\begin{callout}[note]
This is \textbf{Note}-style content, similar to Obsidian \texttt{::: note} or \texttt{> [!note]}.
It can contain multiple paragraphs and lists.
\begin{itemize}[noitemsep]
\item First item
\item Second item
\end{itemize}
\end{callout}

\begin{callout}[tip][Custom title]
Tip with a custom title:
use \texttt{tcolorbox} and set \texttt{title=\{\#1\}}.
This matches Obsidian callout title behaviour.
\end{callout}

\begin{callout}[warning]
Warning without custom title; the default ``Warning'' is shown. \lipsum[1][1-2]
\end{callout}

\begin{callout}[info][Info caption]
Info box for explanatory content. Multi-column layout is in the next section.
\end{callout}

\section{Multi-column (multicol)}

\begin{multicols}{2}
\subsection*{Column one}
\lipsum[1][1-4]

\subsection*{Column two}
\lipsum[2][1-4]
\end{multicols}

\section{Multi-column (minipage, custom ratio)}

% 通过 minipage 比例模拟 ::: col|width(...) 的自定义分栏。
% 0.58 + 0.38 + \hfill 间距 ~= 100% 可用宽度。
\noindent
\begin{minipage}[t]{0.58\textwidth}
\subsection*{Left (58\%)}
\lipsum[3][1-3]
\end{minipage}\hfill
\begin{minipage}[t]{0.38\textwidth}
\subsection*{Right (38\%)}
\lipsum[4][1-3]
\end{minipage}

\vspace{1em}

\section{Callout + columns}

% 使用 paracol 做稳定的双栏排版,避免 tcbraster 在某些模板下退化为竖排。
\columnratio{0.7,0.3}
\begin{paracol}{2}
\begin{callout}[tip][]
Using callouts inside a column
mimics Obsidian @col blocks.
\end{callout}
\switchcolumn
\begin{callout}[note][Right]
Placing note or warning boxes in each column
creates a highlighted layout.
\end{callout}
\end{paracol}


\end{document}